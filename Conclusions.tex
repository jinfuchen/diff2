\section{Conclusions} \label{sec:conclusions}

In this paper, we propose to leverage automated approaches
that can effectively detect early performance regressions in the field. 
We study three approaches that use black-box performance models to capture the relationship between the activities of a software system and its performance under such activities, and then compare the black-box models derived from the current version of a software system and an earlier version of the same software system, to determine the existence of performance regressions between these two versions.
By an empirical study on two open source systems (OpenMRS and Apache James) and one large-scale industry system, we found that simple black-box models (e.g., random forest) can accurately capture the relationship between the performance of a system under varying workloads and its dynamic activities that are recorded in logs.
We also found that these black-box models can effectively detect real performance regressions and injected performance regressions under varying workloads, requiring data from only a short period of operations.
Our approaches can complement or even replace typical in-house performance testing when testing resources are limited (e.g., in an agile environment).
The challenges and the lessons that we learned from the successful adoption of our approach also provide insights for practitioners who are interested in performance regression detection in the field.