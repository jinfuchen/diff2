\section{Approaches} \label{sec:approach}


In this section, we present our three approaches that automatically detect performance regression in a new version of a software system based on the logs and performance metrics that are collected under varying workloads. Our approaches contain three main steps: 1) preparing data, 2) building black-box performance models, and 3) detecting performance regressions. The three approaches share the first two steps, while being different in the third step. The overview of our studied approaches is shown in Figure~\ref{fig:overview}.

\subsection{Preparing data}
We aim to identify whether there is performance regression in the new version of the system based on modeling the relationship between the performance metrics (e.g., CPU usage) that are recorded during system execution and the corresponding logs that are generated during system execution.

\subsubsection{Splitting data into time periods}
We would like to establish the relationship between the execution of the system and the performance of the system during run time. Since both performance metrics and logs are generated during system runtime and are not synchronized, i.e., there is no corresponding record of performance metrics for each line of logs, we would first align the logs and records of performance metrics by splitting them into time periods. For example, one may split a  two-hour dataset into 120 time periods where each time period is one minute in the data. Each log line and each record of performance metric are allocated into their corresponding time period. Then, we consider the aggregation (e.g., taking the average) of the records of performance metrics as the value of the performance metric of the time period, similar to prior research~\citep{Foo:2010:MPR:1848650.1849222}. 

\subsubsection{Extracting log metrics}
We collect logs that are readily generated during the execution of software systems. Such logs represent the workload of the system during a period of execution. We then calculate log metrics based on those logs. In particular, we parse the collected logs into events and their corresponding time stamps. For example, a line of web log ``\emph{[2019-09-27 22:43:13] GET /openmrs/ws/rest/v1/person/ HTTP/1.1 200}" will be parsed into the corresponding web request or URL ``\emph{GET /openmrs/ws/rest/v1/person/}" and time stamp ``\emph{2019-09-27 22:43:13}". Afterward, each value of a log metric is the number of times that each log event executes during the period. For example, if the log event ``\emph{GET /openmrs/ws/rest/v1/person/}" is executed 10 times during a 30-second time period, the corresponding log metric for ``\emph{GET /openmrs/ws/rest/v1/person/}"’s value is 10 for that period. 

\begin{figure*}[tbh]
  \centering
  \includegraphics[width=\textwidth]{overview.pdf}
  \caption{An overview of our studied approaches of detecting performance regressions.}
  \label{fig:overview}
\end{figure*}

\subsection{Building black-box performance models}
\label{sec:buildmodel}
In this step, we build black-box models based on the log metrics and performance metrics that are collected and calculated from the data that is generated during system runtime under varied workloads. 

\subsubsection{Reducing metrics}
The frequency of some log events (e.g., periodical events) may not change over time.
The constant appearance of such events may not provide information about the changes in system workloads.
Therefore, after we calculate the log metrics of each log event, we reduce log metrics by removing redundant log metrics or log metrics with constant values in both previous and current versions. We first remove log metrics that have zero variance in both versions of the performance tests. 

Different log events may always appear at the same time, e.g., user logging in and checking user's privilege, and provide repetitive information for the workloads. To avoid bias from such repetitive information, we then perform correlation analysis and redundancy analysis on the log metrics. We calculate Pearson's correlation coefficient~\citep{benesty2009pearson} between each pair of log metrics. If a pair of log metrics have a correlation higher than $0.7$, we remove the one that has a higher average correlation with all other metrics. 
We repeat the process until there exists no correlation higher than $0.7$. The redundancy analysis would consider a log metric redundant if it can be predicted from a combination of other log metrics. We use each log metric as a dependent variable and use the rest of the log metrics as independent variables to build a regression model. We calculate the $R^2$ of each model. If the $R^2$ is larger than a threshold (e.g., 0.9), the current dependent variable (i.e., the log metric) is considered redundant. We then remove the log metric with the highest $R^2$ and repeat the process until no log metrics can be predicted with an $R^2$ higher than the threshold. 

We only apply this step when using traditional statistical models or machine learning models (like linear regression or random forest), while if a deep neural network (like convolutional neural network or recurrent neural network) is adopted to build the black-box performance models, we skip this step.


\subsubsection{Building models}
We build models that capture the relationship between a certain workload that is represented by the logs and the system performance. In particular, the independent variables are the log metrics from the last step and the dependent variable is the target performance metric (e.g., CPU usage). One may choose different types of statistical, machine learning or deep learning models, as our approach is agnostic to the choice of models. However, the results of using different types of models may vary (cf. RQ1). 


\subsection{Detecting performance regressions} \label{sec:comparions-approaches}

The goal of building performance models is to detect performance regressions. Therefore, in this step, we use the black-box performance models that are built from an old version of the system to predict the expected system performance of a new version. Then, we use statistical analysis to determine whether there exists performance deviance based on prediction errors of the models.

Intuitively, one may use the model that is built from the old version of the system to predict the performance metrics from running the new version of the system. By measuring the prediction error, one may be able to determine whether there exists performance deviance~\citep{DBLP:conf/osdi/CohenCGKS04,DBLP:conf/wosp/NguyenAJHNF12}. However, such a naive approach may be biased by the choice of thresholds that are used to determine whether there is performance deviance. For example, a well-built performance model may only have less than 5\% average prediction error; while another less fit performance model may have 15\% average prediction error. In these cases, it is challenging to determine whether an average prediction error of 10\% on the new version of the system should be considered as a performance regression. Therefore, statistical analyses are used to detect performance regressions in a systematic manner~\citep{DBLP:conf/icst/GaoJBL16,DBLP:conf/wosp/ShangHNF15,Foo:2015:ICS:2819009.2819034}. 

In particular, we leverage three approaches to detect performance regressions: \emph{Approach 1}) by comparing the predicted performance metrics (using the model built from the old version) and the actual performance metrics of the new version, \emph{Approach 2}) by comparing the predicted performance metrics of the new version using the model built from the old version and the model built from the new version, and \emph{Approach 3}) by comparing the prediction errors of the performance metrics on the new version using the model built from the old version and the model built from the new version. We describe each approach in detail in the rest of this subsection.

\noindent\textbf{Approach 1: }\emph{comparing the predicted performance metrics (using the model built form the old version) and the actual performance metrics of the new version.} %\hfill
The most intuitive way of detecting performance regression is to compare the predicted value and the actual value of performance metrics. Since the model is built from the old version of the system, if there exists a large error between the predicted value and the actual value of the performance metrics from the new version of the system, we may consider the existence of performance regressions. In particular, we use the data from the old version of the system, i.e., $Data_{old}$ to build a black-box performance model $Model_{old}$ (cf. Section~\ref{sec:buildmodel}). Afterward, we apply the model $Model_{old}$ on the data from the new version of the system, i.e., $Data_{new}$. Then we compare the predicted and the actual values of the performance metrics.


\noindent\textbf{Approach 2: }\emph{comparing the predicted performance metrics of the new version using the model built from the old version and the model built from the new version.} %\hfill
Since our approach aims to be applied to varying workloads, the performance regression may only impact a small number of time periods, while the source code with performance regressions may not be executed in other time periods. Therefore, only the time periods that are impacted by the performance regressions may contain large prediction errors. To address such an issue, we also built a performance model $Model_{new}$ using the data from the new version of the system ($Data_{new}$). 
This way, $Model_{new}$ is built using the data with potential performance regressions. 
Afterwards, we use $Model_{old}$ and $Model_{new}$ to predict the performance metrics in $Data_{new}$. 

Since $Model_{new}$ is built from $Data_{new}$, there exist a bias when applying $Model_{new}$ on $Data_{new}$. To avoid the bias, instead of building one $Model_{new}$, we build $n$ models, where $n$ is the number of data points that exist in $Data_{new}$. In particular, for each data point in $Data_{new}$, we build a performance model $Model_{new}^n$ by excluding that data point and apply the model $Model_{new}^n$ on the excluded data point. Therefore, for $n$ data points from $Data_{new}$, we end up having $n$ models and $n$ predicted values. 

Finally, we compare the predicted values using $Model_{old}$ on $Data_{new}$, and the predicted values by applying each $Model_{new}^n$ on each of the $n$ data points in $Data_{new}$.


\noindent\textbf{Approach 3: }\emph{comparing the prediction errors on the new version using the model built from the old version and the model built from the new version.}
The final approach of detecting performance regression is similar to the previous one (Approach 2), where instead of directly comparing the predicted performance metrics, we compare the distribution of the prediction errors by applying $Model_{old}$ on $Data_{new}$, and the predicted errors by applying each $Model_{new}^n$ on each of the $n$ data points in $Data_{new}$. The intuition is that when $Model_{old}$ has a larger prediction error than $Model_{new}$, it may be an indication of performance deviance between the two versions. We note that this approach would only be able to determine whether there exists performance deviance, which may actually be improvement instead of regression. 


\noindent\textbf{Statistical analysis.}
All three approaches generate two distributions of either actual/predicted performance metric values, or prediction errors. With two distributions of data at hand, we compare the two distributions similar to previous studies~\citep{Chen:2016:CHD:2950290.2950303} using statistical tests and effect sizes. In particular, we use the Mann-Whitney U test since it is non-parametric and it does not assume a normal distribution of the compared data. we run the test at the 5\% level of significance, i.e., if the P-value of the test is not greater than 0.05, we would reject the \emph{null hypothesis} in favor of the alternative hypothesis, i.e., there exists a statistically significant difference between the two distributions. In order to study the magnitude of the difference without being biased by the size of the data, we further adopt the effect size as a complement
of the statistical significance test. Considering the non-normality of our data points, we
utilize \emph{Cliff's Delta}~\citep{cliff1996ordinal} using the
thresholds provided in prior research~\citep{romano2006appropriate}.


