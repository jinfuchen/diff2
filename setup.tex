% ------------- Packages Used --------------------------------------------------
% They help us to produce a better looking document ;-)
% ------------------------------------------------------------------------------

% Comment the following line before compiling the final version
%\synctex
\usepackage{multirow}
\usepackage{balance}
\usepackage{graphicx}
\usepackage{alltt}
\usepackage{relsize}
%\usepackage{xspace}
\usepackage{booktabs}
\usepackage{array}
\usepackage{amsmath}
%\usepackage{multirow}
%\usepackage{array}
\usepackage{verbatim}

%\usepackage{subfigure}
\usepackage[tight,footnotesize]{subfigure}

%\usepackage{capt-of}
%\usepackage{pifont}
\usepackage{amsfonts}
\usepackage{amssymb}
%\usepackage[latin1]{inputenc}
%\usepackage{times}
%\usepackage{colortbl}
\usepackage{boxedminipage}
\usepackage{float}
\usepackage{cite}
\usepackage{fancyvrb}
\usepackage[dvipsnames]{xcolor}
%\usepackage{hyperref}
\usepackage{balance}
\usepackage{url}
\usepackage{fancybox}%for \hypobox
\usepackage{listings}
\usepackage{array}
\usepackage{lscape}



%\usepackage{textcomp}
%\usepackage{latexsym}
%\usepackage{amssymb}
%\usepackage{stmaryrd}
%\usepackage{euscript}
%\usepackage{wasysym}
%\usepackage{pifont}
%\usepackage{manfnt}
%\usepackage{undertilde}
%\usepackage{ifsym}
%\usepackage{tipa}
%\usepackage{txfonts}
%\usepackage{skak}
%\usepackage{skull}
%\usepackage{eurosym}
%\usepackage{yfonts}
%\usepackage{mathdots}
%\usepackage{trsym}
%\usepackage{upgreek}
%\usepackage{chemarr}
%\usepackage{accents}
%\usepackage{nicefrac}
%\usepackage{bm}


%\usepackage{pdfsync}

%   ACM Style
%\usepackage{lcsect}
% ------------------------------------------------------------------------------






% ------------ Color Definitions -----------------------------------------------
% Whatever colors we need
% ------------------------------------------------------------------------------
\definecolor{mygray}{rgb}{0.7,0.7,0.7}
% ------------------------------------------------------------------------------






% ---------- Special commands for annotating the paper's text ------------------
\let\mymarginpar\marginparm
\marginparwidth=1cm
\marginparsep=5pt
\newcommand{\todo}[1]{\textcolor{red}{\textbf{[[#1]]}}}
\def\TODO#1{\noindent\colorbox{yellow}{\bf \textcolor{red}{TODO: #1}}}
\newcommand{\hint}[1]{\textcolor{blue}{\textbf{#1}}}
\def\fig#1{Figure~\ref{#1}}
\def\tab#1{Table~\ref{#1}}
\def\eqn#1{Equation~\ref{#1}}
\def\sec#1{Section~\ref{#1}}

\pagenumbering{arabic}
\newcommand{\reviewer}[1]{\textcolor{DeepPink1}{{\it [Reviewer says: #1]}}}
\newcommand{\ian}[1]{\textcolor{blue}{{\it [Ian says: #1]}}}
\newcommand{\heng}[1]{\textcolor{blue}{{\it [Heng says: #1]}}}
\newcommand{\ahmed}[1]{\textcolor{red}{{\it [Ahmed says: #1]}}}
\newcommand{\myfoot}[1]{\footnote{\scriptsize #1}}
\newcommand{\myurl}[1]{\myfoot{\url{#1}}}
\newcommand{\Ra}{{$\Rightarrow$}}
\newcommand{\ra}{{$\rightarrow$}}
\newcommand{\La}{{$\Leftarrow$}}
\newcommand{\la}{{$\leftarrow$}}
\newcommand{\lra}{{$\leftrightarrow$}}
\newcommand{\LRa}{{$\Leftrightarrow$}}
%\newcommand{\todo}{\bram{todo}}

\newenvironment{myindentpar}[1]%
{\begin{list}{}%
         {\setlength{\leftmargin}{#1}}%
         \item[]%
}
{\end{list}}

% \AtBeginDocument{%
%    \renewcommand{\figurename}{Figure}%
%    \newcommand{\subfigureautorefname}{\figureautorefname}%for using subfig
% %   \renewcommand{\tablename}{TABLE}%
%    \renewcommand{\tablename}{Table}%
%    \renewcommand{\subsectionautorefname}{Section}%
%    \renewcommand{\sectionautorefname}{Section}%
% }

% Hypothesis box	
% ------------------------------------------------------------------------------	
\newcommand{\hypobox}[1]{\begin{center}%	
	\noindent\thicklines\setlength{\fboxsep}{7pt}%	
	%\cornersize{0}\Ovalbox{\begin{minipage}{3.2in}%
	\cornersize{0}\Ovalbox{\begin{minipage}{4in}%	
	\vspace{-0.1cm}
	\textit{#1}
	\vspace{-0.1cm}
	\end{minipage}} \end{center}}	
% ------------------------------------------------------------------------------




% ------------------------- SYMBOLS OF SELF NAMES OFTEN USED -------------------
\newcommand{\APACHE}{{\small APACHE}\xspace}
\newcommand{\BUGZILLA}{{\small BUGZILLA}\xspace}
\newcommand{\ECLIPSE}{{\small ECLIPSE}\xspace}
\newcommand{\ASPECTJ}{{\small ASPECTJ}\xspace}
\newcommand{\JDT}{{\small JDT}\xspace}
\newcommand{\GNU}{{\small GNU}\xspace}
\newcommand{\MOZILLA}{{\small MOZILLA}\xspace}
\newcommand{\THUNDERBIRD}{{\small THUNDERBIRD}\xspace}
\newcommand{\JAVA}{{\small Java}\xspace}
\newcommand{\GNOME}{{\small GNOME}\xspace}
\newcommand{\PG}{{\small PostgreSQL}\xspace}
\newcommand{\SIM}{{\small SimScan}\xspace}

% Anything else, e.g., \NAME{MICROSOFT}
\newcommand{\NAME}[1]{{\small #1}\xspace}
% ------------------------------------------------------------------------------




% ----------------------- Computer Science lol ---------------------------------
% Variable, function, and program names
% ------------------------------------------------------------------------------
\newcommand{\smalltt}[1]{\ifmmode{\mbox{\smaller\texttt{#1}}}\else{\smaller\tt #1}\fi}
\newcommand{\code}[1]{\smalltt{#1}}
\newcommand{\var}[1]{\code{#1}}
\newcommand{\func}[1]{\code{#1}}
\newcommand{\proc}[1]{\code{#1}}
\newcommand{\prog}[1]{\code{#1}}
\newcommand{\type}[1]{\code{#1}}
\newcommand{\progpt}[1]{\code{#1}}

\newcommand{\mypar}[1]{\vspace{.1cm}\noindent \textbf{#1}}
\newcommand{\myxpar}[1]{\vspace{.1cm}\noindent \textbf{#1}\newline}
% ------------------------------------------------------------------------------





% ----------- Things to remember -----------------------------------------------
\newenvironment{mynote}%
{ \medskip
  \noindent
  \let\emph=\textbf
  \begin{boxedminipage}{\columnwidth}\em}%
{ \end{boxedminipage}}
% ------------------------------------------------------------------------------






% -------------------- Use bars ------------------------------------------------
% These macros are for advanced presentation of results by shaded bars only!
% © Tom Zimmermann, 2008
% ------------------------------------------------------------------------------
\newdimen\qdx
\newdimen\qda
\newdimen\qdb
\def\rrrr#1#2#3#4{\newdimen\qd\qd=#4 % length of bar for 1.0
\qdx=\qd\multiply\qdx by 5\divide\qdx by 4
\qda=\qd
\qdb=\qd
\multiply\qda by #1\divide\qda by #3\multiply\qdb by #2\divide\qdb by #3\advance\qdb by -\qda
    \leavevmode\hbox to \qdx{\hfil\vbox{%
    \hbox{\vrule\vbox{\hrule\hbox to 1\qd
            {\vrule depth0pt height0.7ex width \qda\color{mygray}%
 \vrule depth0pt height0.7ex width \qdb\hfill}\hrule}\vrule}
    }\hfil}}
\def\rrr#1#2#3{\rrrr{#1}{#2}{#3}{0.8cm}}
% -------------------------------------------------------------------------






% ------------ Graphics Hacks --------------------------------------------------
% Use these settings if Figures tend to get their one separate pages
% ------------------------------------------------------------------------------
% \renewcommand{\topfraction}{0.85}
% \renewcommand{\textfraction}{0.1}
% \renewcommand{\floatpagefraction}{0.75}
% ------------------------------------------------------------------------------





% ------------------ Biblio Hack -----------------------------------------------
% Use these settings if the References take too much space
% ------------------------------------------------------------------------------
\let\oldthebibliography=\thebibliography
  \let\endoldthebibliography=\endthebibliography
  \renewenvironment{thebibliography}[1]{%
%	\vspace{-0.3cm}
    \begin{oldthebibliography}{#1}%
%	\vspace{-0.2cm}
       \setlength{\parskip}{0ex}%
       \setlength{\itemsep}{0ex}%
%	\bibfont
  }%
  {%
    \end{oldthebibliography}%
  }
% ------------------------------------------------------------------------------





% -------------- Floats Redefined ----------------------------------------------
% Use these settings to change the whitespace between floats and text
% ------------------------------------------------------------------------------

% \setlength\dblfloatsep{1pt}
% \setlength\floatsep{1pt}
% \setlength\textfloatsep{5pt}
% \setlength\dbltextfloatsep{5pt}

% \renewcommand\floatpagefraction{.9}
% \renewcommand\topfraction{.9}
% \renewcommand\bottomfraction{.9}
% \renewcommand\textfraction{.1}
% \setcounter{totalnumber}{50}
% \setcounter{topnumber}{50}
% \setcounter{bottomnumber}{50}
% ------------------------------------------------------------------------------
